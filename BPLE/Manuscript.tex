% an empty .tex file. Eventually may transfer .odt in here
%%This is a very basic article template.
%%There is just one section and two subsections.
%\documentclass[12pt,oneside,a4paper,doublespacing]{article} % for submission
\documentclass[11pt,oneside,a4paper]{article} % for sharing

\usepackage{appendix}
\usepackage{amsmath}
\usepackage{caption}
\usepackage{placeins}
\usepackage{graphicx}
\usepackage{subcaption}
%\usepackage{subfig}
\usepackage{longtable}
\usepackage{setspace}
%\usepackage{tikz}
\usepackage{booktabs}
\usepackage{tabularx}
\usepackage{xcolor,colortbl}
\usepackage{chngpage}
%\usepackage[active,tightpage]{preview}
\usepackage{natbib}
\bibpunct{(}{)}{,}{a}{}{;} 
\usepackage{url}
\usepackage{nth}
\usepackage{authblk}
\usepackage[most]{tcolorbox}
%\usepackage{hyperref}
%\usepackage{color}
%\usepackage{fontspec}
%\usepackage{pdfsync}
\usepackage[normalem]{ulem}
\usepackage{amsfonts}
\renewcommand{\listtablename}{List of Appendix Tables}
\newcolumntype{C}[1]{>{\centering\let\newline\\\arraybackslash\hspace{0pt}}m{#1}}
\newcolumntype{L}[1]{>{\raggedright\let\newline\\\arraybackslash\hspace{0pt}}m{#1}}
% working on this need to concatenate file name based on sex and variable name
%\newcommand\Cell[1]{{\raisebox{-0.05in}{\includegraphics[height=.2in,width=.2in]{Figures/ColorCodes/\expandafter#1}}}}  

%%%%%%%%%%%%%%%%%%%%%%%%%%%%%%%%%%%%%%%%%%%%%%%%%%%%%%%%%%%%%%%%%%%%%%%%%%%%%
% setting color to letters affects spacing. Here's a hack I found here:
% http://tex.stackexchange.com/questions/212736/change-letter-colour-without-losing-letter-spacing
%\DeclareRobustCommand{\spacedallcaps}[1]{\MakeUppercase{\textsc{#1}}} % all
% caps with better spacing

%\colorlet{RED}{red}
%\colorlet{BLUE}{b}
\colorlet{rd}{red}
\colorlet{bl}{blue}

%%%%%%%%%%%%%%%%%%%%%%%%%%%%%%%%%%%%%%%%%%%%%%%%%%%%%%%%%%%%%%%%%%%%%%%%%%%%%%

\newcommand\ackn[1]{%
  \begingroup
  \renewcommand\thefootnote{}\footnote{#1}%
  \addtocounter{footnote}{-1}%
  \endgroup
}
\newcommand\vt[1]{\textcolor{rd}{#1}}
\newcommand\eg[1]{\textcolor{bl}{#1}}

\newcommand\tg[1]{\includegraphics[scale=.5]{Figures/triadtable/triad#1.pdf}}
\newcommand\tgh[1]{\raisebox{-.25\height}{\includegraphics[scale=.3]{Figures/triadtable/triad#1.pdf}}}

\defcitealias{HMD}{HMD}

% junk for longtable caption
\AtBeginEnvironment{longtable}{\linespread{1}\selectfont}
\setlength{\LTcapwidth}{\linewidth}

%%%%%%%%%%%%%%%%%%%%%%%%%%%%%%%
\begin{document}

\title{A best practices composite lifetable for the United States}

\author[1]{Tim Riffe\thanks{triffe@demog.berkeley.edu}}
\author[2,3]{Adrien Remund}
\author[3,4]{Magali Barbieri}
\author[4]{Celeste Winant}
\affil[1]{Max Planck Institute for Demographic Research}
\affil[2]{Universit{\'e} de Gen{\`e}ve}
\affil[3]{Institut national d'{\'e}tudes d{\'e}mographiques}
\affil[4]{Department of Demography, University of California, Berkeley}

%\author{[Authors]}

\maketitle

\begin{abstract}
We calculate a preliminary series of best-possible lifetables for the United
States from 1959 to 2004, defined on the basis of the age-specific aggregate of
the lowest observed age-cause specific death rates among the 50
states and the District of Columbia for each year. The resulting highest-possible life expectancy shows a gradual increasing trend over the period, on
average 1.9 and 2.2 years higher than the highest state life expectancy in each year for males and females, respectively. We argue that the US states best practices lifetable is a useful
gauge of future mortality trends and a good benchmark against which to compare
state mortality patterns.
\ackn{The work reported in this manuscript was supported in part by the U.S.
National Institute On Aging of the National Institutes of Health under award
numbers R01-AG011552 and R01-AG040245. The content is solely the responsibility of the authors and does not necessarily represent the official views of the funding agencies.}
\end{abstract}


\section*{Introduction}

The question of limits to life expectancy is fundamental to demography, but it
is also practical when projecting mortality. Mortality reductions in past
decades have been steady in many populations, and at times linear, which tends
to guide projections into predicting the same sort of progress far into the
future. Most past attempts to place limits on such projections were shown to be
overly conservative within a short period \citep{oeppen2002broken}.
However, many simple mortality projections will send life expectancy scenarios
to very high levels that for a given population may seem unrealistic. In some
cases, one seeks an external population that has already achieved a life
expectancy as high as that produced by a model, and this gives some assurance
that the projection is possible. Japan assumes this role in many cases today. In
2012 life expectancy for Japanese females was 86, while for US females it was 81
\citep{HMD}. If a US projection turns out to reach 86 we at least know that
there exists some set of conditions in the world today that imply that this is
indeed possible.

The desire for such external references explains part of the interest in the
historical development of world record life expectancy \citep{oeppen2002broken}.
The maximum life expectancy from a set of populations shows what other
populations may one day achieve due to diffusion in practices, technology, and
wellbeing \citep{vallin2010esperance}. The highest observed life expectancy in
a given moment, which we term the ``vanguard'' life expectancy, is calculated
based on mortality rates undifferentiated by cause of death. All cause mortality may not provide an omnibus signal of what may eventually develop. While the force of mortality governs the lifetable, there is a substantive rationale to conceive of
this force as a composite of cause-specific forces of mortality.
\citet{vallin2008minimum} separate trends in life expectancy by causes of death,
explained in terms of an epidemiological transition that unfolds in progressive
stages with respect to particular technologies and risk factors in recent
history. 

The level and timing of cause-specific responses to particular
technological or well-being improvements varies between populations. Therefore,
the lowest observed all-cause force of mortality is not necessarily composed of
the lowest cause-specific forces of mortality, which means that the vanguard life
expectancy may not reflect the best mortality possible given current conditions
over a set of populations. For instance, the lowest rate of suicide and heart
disease at age 45 will not necessarily be found in the same population. As an
alternative to vanguard life expectancy, one could create cause-specific
low-mortality benchmarks, or more synthetically, combine the lowest mortality
observed by age and cause into a hybrid minimum mortality schedule. This synthetic low mortality lifetable--- hybrid with respect to reference population over age and cause--- was already suggested by \citet{wunsch1975minimum} and \citet{vallin2008minimum}. These authors investigated trends at the national-level. We refer to such hybrid minimum
lifetables as ``best practices'' lifetables. \citet{eikemo2014} use the term
``best practices'' in a similar sense of an ideal mix of condtions, but do so via
combining particular risk factors with their estimated mortality impacts to
produce best case scenarios. The end effect is quite similar, to the extent that
either estimation strategy shows what society could acheive given perfect
diffusion of some defintion of best conditions. In the case of a best practices
lifetable, we condition on mortality outcomes only, making the endeavor much
simpler in practice.

National populations are heterogeneous with respect to many factors, both
observed and unobserved, that affect cause-spectific mortality rates. This does
not make international mortality comparisons futile, but it may make perfect
convergence unimaginable even in the very long run. This is the case for both
all cause and cause-specific mortality. When projecting mortality for a given country, we wonder whether mortality from an external population, or set of populations, is the best ``sanity
check'' for mortality forecasts.
For this reason, we propose a minimum mortality trend for the United States
calculated on the basis of its own territorial subdivisions, the 50 states plus
the District of Columbia. This choice limits the sources of unaccounted
heterogeneity in international comparisons, and it also guarantees uniformity of
data sources.
To be clear, we do not necessarily propose using a trend in best practices mortality as a component of coherent mortality forecasts, and we do
not undertake such an exercise. Instead, we view the U.S. best practices
mortality as indicative of the potential already present in the U.S., a special
kind of ``nowcast''.

\section*{Data and methods}
All death count data come from the public NCHS mortality microdata files
\citep{NCHSdata}.\footnote{The current analysis is based on the public files
with geographic information only up to 2004. We have bene granted access to
geographic variables in a NCHS Research Data Center, and will extend analysis
through 2013 for the final paper version.} Population denominators come from a
beta version of the U.S.
States project of the Human Mortality Database \citep{HMD}. Further processing of
population counts into exposures is done according to the HMD Methods
Protocol, and all-cause lifetables were calculated by single ages \citep{HMDMP}.
Death counts for causes were aggregated into quinquennial age groups by year and sex for eleven large groups that minimize disruptions between ICD transitions, and then
converted to fractions. Fractions were then multiplied into the single-age
all-cause death rates. The eleven causes of death considered in this abstract
version are displayed in Table~\ref{tab:desc}, with their relative percentages.

% latex table generated in R 3.1.2 by xtable 1.7-4 package
% Wed Sep 23 17:46:55 2015
\begin{table}[ht]
\centering
\caption{Eleven causes of death considered in the current study version,
percentages.}
\label{tab:desc}
\begin{tabular}{rrr}
 & Females & Males \\ 
  \hline
All other & 29.91 & 33.95 \\ 
  Breast & 3.73 & 0.03 \\ 
  Cardiovascular & 38.49 & 38.10 \\ 
  Cerebrovascular & 10.50 & 6.84 \\ 
  Lung & 3.55 & 6.50 \\ 
  Other Cerebrovascular & 0.33 & 0.28 \\ 
  Other Malignant Neoplasms & 11.01 & 9.83 \\ 
  Other smoking & 0.57 & 1.36 \\ 
  Prostate &  & 2.25 \\ 
  Stomach & 0.67 & 0.87 \\ 
  Uterus & 1.25 &  \\ 
   \hline
\end{tabular}
\end{table}

One caveat of calculating a best practices lifetable is that stochastic zeros
occur in particular ages and causes, especially among small populations. The
exercise of selecting the minimum death rate for a given age and cause will tend
to collect stochastic zeros, producing an excessively optimistic best practices
life expectancy. We therefore smooth all death rates using
the \texttt{MortalitySmooth} package \citep{GC2012} in \texttt{R}. The age-period
matrix of death rates for each state and sex is smoothed over both margins
using a penalized b-spline. Cause fractions are then recalculated, and
multiplied into the smoothed all-cause death rates. Finally, we calculate
lifetables for each sex and state, as well as the best practices lifetable of
each year using HMD methodology.

We decompose differences in state life expectancy from the best-practices life
expectancy using the pseudo-continuous method proposed by
\citet{horiuchi2008decomposition}. We prefer this method because it allows for
defining life expectancy as a direct function of age-cause-specific rates,
$M_{(x,c)}$, thereby eliminating some artifacts present in other methods.
Results are summarized by ages, causes and states and can be interpreted
directly as the the total lag in years of life expectancy due to excess
mortality (anything greater than the minimum) in the given margin.\footnote{In
the final version we may opt to further separate trends from initial
differences using the decomposition method recently proposed by
\citet{DimaDecomp2014}.}

\subsection*{Vanguard and best practices life expectancy}
For the sake of clarity, we define vanguard life expectancy as the highest
observed state life expectancy for each sex in a given year. Elsewhere, this
quantity is called the record life expectancy \citep{oeppen2002broken}, or even
the best practices life expectancy \citep{sanderson2004putting}, but we apply
the term best practices in a different sense. To define the best practices
mortality schedule, select the lowest (smoothed) death rate by age, $x$,
and cause, $c$, from among all the states, $s$, $m(s,x,c)$, and sum over
all causes, $C$, within ages to define the minimum mortality schedule,
$m(x)^{min}$
\begin{equation}
m(x)^{min} = \sum _{c=1}^C min(m(s,x,c))
\end{equation}

The minimum mortality schedule is then used to calculate the best-practices life
expectancy, and its cause-specific mortality pattern is used as the
benchmark against which to decompose the mortality pattern of each state. While
the best practices life expectancy is drawn from the diverse mortality
experience of the entire USA, the vanguard life expectancy refers only to a
particular state in a given year. We can say, for instance, that Minnesota was
the vanguard for females in the year 1959, with a lif expectancy of 75.3, but
that the best practices life expectancy in the same year was another 3 years
higher at 78.3.

There are limitations to the interpretation of a best practices trend, as
defined here. The mix of causes within each age, and the resulting age pattern
for each cause may not necessarily reflect a possible or likely mortality
pattern. For instance, diseases may compensate each other, especially for
diagnostic and reporting reasons. For instance, for decedents that suffered
from both condition A and condition B, one state might be more likely to
declare A as the underlying cause of death, while another state would more
likely declare B to be as the underlying cause. A situation like this
would tend to inflate the best practices $e(0)$. The best practices $e(0)$ may
be underestimate in instances where reductions in cause A are expected to lead
to reductions in cause B (e.g., pneumonia and cardiovascular disease.)
\FloatBarrier
\section*{Preliminary results}
\FloatBarrier

All results at this time are preliminary, but we can report broad trends.\footnote{The final results will differ in the following ways. At this time we have
calculated results using 11 causes of death that were previously from a
different research project, and this will change to 30 causes to be consistent and comparable with \citet{vallin2008minimum}. Including
more causes will increase the best practices life expectancy level. We will also
include estimates through 2012 or 2013. At this time, we were only able to
produce results through 2004, because geographic information is suppressed in
the public microdata for later years. This information is available in secure
Research Data Centers, where we will produce cause fractions and all-cause lifetables for
the most recent decade.} 

\subsection*{Life expectancy trends}
Most states display irregular gradual increasing trends
over the 45 years considered, except for males in the District of Columbia,
which have been recovering strongly since the 1990s, due both to secular
improvements and compositional change. The US aggregate trend is bouncier
because death counts were not smoothed, and its rank position among states
varies due in part also to compositional changes between states. The vanguard
state life expectancy for males is Hawaii until the two most recent years of
observation (2003-2004), where Minnesota has taken the lead. For females,
Minnesota was vanguard until 1963, followed by North Dakota for five years, and
then Hawaii for the rest of the period until 2004. The vanguard trend for
females has the same overall shape as the cluster of individual state trends, but for males the time trend in vanguard, ergo Hawaiian, life expectancy was
different from the bulk of states. 

For both males and females, the best
practices trend is on average 2 years higher than the vanguard trend, and its
progress is steadier than either the vanguard, national, or average state
trends, as well as most individual states. The best practices trend has increased each year for
males and females, though faster for males. The average annual improvement for
males has been 0.1790 years of $e(0)$ per annum, compared with 0.1785 for the
male national average life expectancy. The respective rates of improvement for
females have been 0.1470 and 0.1456. It is convenient and telling that the
respective medium term average rates of improvement between national and
best practices life expectancy match so closely, despite the best practices
trend having a much more stable trajectory. This artifact may provide some
support to conceiving of the best practices trend as indicative of where the
U.S. is going in the coming decades.

\begin{figure}[t!]
\caption{Trends in US, vanguard, best practices, and state life expectancies at
birth ($e(0)$), 1959-2004.}
\label{fig:e0}
\centering
\begin{subfigure}[b]{.48\textwidth}
\centering
\caption{Males}
\label{fig:e0m}
\includegraphics[scale=0.5]{Figures/e0trendsM.pdf}
\end{subfigure}
~
\begin{subfigure}[b]{.48\textwidth}
\centering
\caption{Females}
\label{fig:e0f}
\includegraphics[scale=0.5]{Figures/e0trendsF.pdf}
\end{subfigure}
\end{figure}

If we take the mean rate of improvement for national or best practice trends,
and compare it with the means of each series, we can roughly conclude that the
best practices trend in year $t$ is a decent estimate of the national life
expectancy in 30 or so years. This time lag is valid for the present series,
but will vary depending on the number of cause groups that enter into the
selection of minimum mortality rates. Clearly, if the population units were
counties rather than states, the level of best practices life expectancy would
increase even further, but we opine that cause-specific smoothing at the county
level is more of a modelling gesture than sound practice, due to very low counts of deaths in
most counties in the U.S. In the final paper we will aim to test the cause set
proposed by \citet{vallin2008minimum} and compare it with other reasonable
configurations for the U.S., and so recommend a rule of thumb grouping. 

For the sake of comparison with the presentlt calculated best practices series.
The $e(0)$ value in 2004 for males and females was 80 and 84.9, respectively. In
the same year, the observed all-cause $e(0)$ in Japan were 78.6 for males and
85.5 for females. That is, the U.S. best practices trend is ahead of observed
life expectancy for Japanese males, and behind for Japanese females. We may also
compare with the official $e(0)$ projections of the U.S. Social Security
Administration (SSA). The SSA projects projects male $e(0)$ passing 80 in the
year 2058, and female $e(0)$
passing 84.9 in the year 2075 \citep{SSA2005}. These projections are a full 54
and 71 years after 2004, in both cases much farther in the future than a simple extrapolation
of the best practices trend would predict, circa 2034.

\FloatBarrier
\subsection*{State differences by cause}

Since the trend in the best practices $e(0)$ is so stable over time, it
is also a convenient moving benchmark against which to compare state patterns
--- a kind of potential mortality for each state. We
decompose the state departures in life expectancy from the best practice trend in each year and sex, over both age and cause. Part of the attraction of decomposing with respect to the best practices life expectancy is that contributions are always of the same sign, which makes heatmaps more intuitive than would otherwise be the case.\footnote{Typically age and/or cause-specific contributions to a difference in $e(0)$ are represented in stacked bar charts, which in the present case would not be a practical way to distill information.} We therefore summarize state trends in a series of heatmaps representing the 50 states plus the District of Columbia. In these heat maps, states are represented as identically-sized square cells, while preserving rough spatial relationships from a standard map projection. For orientation, we present a state guide to the heatmaps in Figure~\ref{fig:heatguide}.\footnote{This configuration is inspired by a tile grid map on the
NPR blog \citep{NPRsquares}, which
itself was inspired by various commonly used maps in the New York Times,
Washington Post, and others. DeBelius notes that grid panels are not meant for
displaying population quantities, since each state has a different population
size. Since lifetable quantities are purged of population size and
structure, the visual instrument is adequate, and the equalization of state
size and form adds legibility to our results.} Deeper shades of red
indicate larger departures, i.e., larger $e(0)$ shortcomings.

\begin{figure}
\centering
\caption{A guide to tiled heatmaps of U.S. states.}
\label{fig:heatguide}
\includegraphics[scale=.3]{Figures/StatesDiagram.pdf}
\end{figure}

In Figures~\ref{fig:depm} and \ref{fig:depf} we represent the average $e(0)$
shortcoming from each cause and state, and average within decades to summarize the total contribution
to the departure from best practices.\footnote{The 2000s results are based on
the average for the years 2000-2004 only.} Major causes of death are in rows,
and decades are organized in columns. For the case of cardiovascular disease and other causes, grouping departures into the 2$+$ group blends out much
interstate variation, since some contributions range as high as 4
years, but broad patterns are still visible.\footnote{This visual display may be
significantly adapted when we add more causes to the decomposition. We will also do a similar small multiples
display for grouped age patterns.}

Glancing over the lung diseases and cancer rows of the maps, it is evident that
the District of Columbia had high excess mortality for these causes over all
five decades, peaking in the 1980s. Lung diseases display a clear mortality
basin in the Southern states for males, although the absolute maginitude of these departures from ideal mortality has diminished over time. The same mortality basin occurs in
cariovascular causes over all five decades, but is clearly visible in this
display only since the 1990's, due to the 2$+$ grouping.

Since the cause groupings available at the time of submission are not the final groupings, and
because the causes included in the study contribute such varying magnitudes to
state differences, trends are dominated by Cardivascular diseases and the large
aggregate group of all other causes. The 11 original causes were grouped into
five so as to make state patterns more visible. These results will be commented
at greater length when the final results are prepared.

\begin{figure}
\centering
\caption{U.S. males, state departures from best practices $e(0)$ by large cause
groups, 1959-2004.}
\label{fig:depm}
\includegraphics[scale=0.7]{Figures/StatesDecadesM.pdf}
\end{figure}

\begin{figure}
\centering
\caption{U.S. feales, state departures from best practices $e(0)$ by large cause
groups, 1959-2004.}
\label{fig:depf}
\includegraphics[scale=0.7]{Figures/StatesDecadesF.pdf}
\end{figure}

\FloatBarrier
\section*{Discussion}
In this paper, we have given an intuitive rationale for calculating a best
practices trend in the United States. First, the best practices trend in life
expectancy at birth can be interpreted as the mortality potential present in the
United States at a given point in time. It is an indication of the mortality
conditions that would come to be if local conditions, practices,
technologies, and behaviors were perfectly diffused from those places with the
best age and cause-specific performance in the year. While it is difficult to
imagine that pressing a magic button to activate this perfect cocktail of
mortality risk factors would actually bring about this minimum mortality
instantaneously--- for instance, cumulative risk factors in population
stocks that are distributed unevenly over space would stymie such a thought
experiment--- we do observe catch-ups with past best practices values for many states, and for the
U.S. national $e(0)$, on average after a period of 30 years. It remains to be
seen how this lag scales to the number of causes ocnsidered and size and
distribution of subpopulations. At this time, we are rather attracted to the
idea of a period weathervane that points us to where period mortality is going.

\bibliographystyle{plainnat}
%\bibliographystyle{demography}
  \bibliography{references.bib}   % Use the BibTeX file ``references.bib''.
\end{document}

